% =============================================================================
% TRABALHO DE PESQUISA OPERACIONAL - NORMAS ABNT
% Otimização de Recursos de Segurança Pública
% =============================================================================

\documentclass[
    12pt,               % Tamanho da fonte
    a4paper,            % Tamanho do papel
    oneside,            % Impressão em um lado
    chapter=TITLE,      % Títulos em maiúsculas
]{article}

% =============================================================================
% PACOTES
% =============================================================================
\usepackage[utf8]{inputenc}
\usepackage{amsmath,amssymb,amsfonts}
\usepackage{graphicx}
\usepackage{float}
\usepackage{booktabs}
\usepackage{tabularx}
\usepackage{setspace}
\usepackage{hyperref}
\usepackage{xcolor}
\usepackage{caption}
\usepackage{titlesec}

% =============================================================================
% CONFIGURAÇÕES ABNT - Margens
% =============================================================================
\usepackage[
    a4paper,
    left=3cm,           % Margem esquerda: 3cm (ABNT)
    right=2cm,          % Margem direita: 2cm (ABNT)
    top=3cm,            % Margem superior: 3cm (ABNT)
    bottom=2cm          % Margem inferior: 2cm (ABNT)
]{geometry}

% =============================================================================
% CONFIGURAÇÕES ABNT - Espaçamento
% =============================================================================
\onehalfspacing         % Espaçamento 1,5 (ABNT)
\setlength{\parindent}{1.25cm}  % Recuo de parágrafo: 1,25cm (ABNT)
\setlength{\parskip}{0pt}       % Sem espaço entre parágrafos

% =============================================================================
% CONFIGURAÇÕES ABNT - Títulos de Seções
% =============================================================================
% Seções primárias: maiúsculas, negrito, tamanho 12
\titleformat{\section}
    {\normalfont\bfseries\normalsize\MakeUppercase}
    {\thesection}{1em}{}

% Seções secundárias: maiúsculas, sem negrito, tamanho 12
\titleformat{\subsection}
    {\normalfont\normalsize}
    {\thesubsection}{1em}{}

% Seções terciárias: negrito, tamanho 12
\titleformat{\subsubsection}
    {\normalfont\bfseries\normalsize}
    {\thesubsubsection}{1em}{}

% Espaçamento antes e depois dos títulos
\titlespacing*{\section}{0pt}{12pt}{6pt}
\titlespacing*{\subsection}{0pt}{12pt}{6pt}
\titlespacing*{\subsubsection}{0pt}{12pt}{6pt}

% =============================================================================
% CONFIGURAÇÕES ABNT - Sumário
% =============================================================================

% =============================================================================
% CONFIGURAÇÕES - Hyperlinks
% =============================================================================
\hypersetup{
    colorlinks=true,
    linkcolor=black,
    citecolor=black,
    urlcolor=blue
}

% =============================================================================
% CAMINHO DAS FIGURAS
% =============================================================================
\graphicspath{{figuras/}}

% =============================================================================
% INÍCIO DO DOCUMENTO
% =============================================================================
\begin{document}

% =============================================================================
% ELEMENTOS PRÉ-TEXTUAIS
% =============================================================================

% -----------------------------------------------------------------------------
% CAPA (Elemento obrigatório - ABNT NBR 14724)
% -----------------------------------------------------------------------------
\begin{titlepage}
    \centering
    \onehalfspacing
    
    % Instituição
    {\normalsize\MakeUppercase{Instituto Federal de Goiás}}\\
    {\normalsize\MakeUppercase{Departamento de Áreas Acadêmicas IV}}\\
    
    \vspace{4cm}
    
    % Autores (ordem alfabética do sobrenome)
    {\normalsize DAVI AUGUSTO BITENCOURT DE SOUZA}\\
    {\normalsize PEDRO LUCAS DUTRA}\\
    
    \vspace{4cm}
    
    % Título
    {\Large\bfseries\MakeUppercase{Otimização de Recursos de Segurança Pública no Brasil:}}\\[0.3cm]
    {\large\MakeUppercase{Aplicação de Programação Linear para Alocação Eficiente de Verbas Estaduais}}\\
    
    \vfill
    
    % Local e data
    {\normalsize Goiânia}\\
    {\normalsize 2026}
\end{titlepage}

% -----------------------------------------------------------------------------
% FOLHA DE ROSTO (Elemento obrigatório - ABNT NBR 14724)
% -----------------------------------------------------------------------------
\begin{titlepage}
    \centering
    \onehalfspacing
    
    % Autores
    {\normalsize DAVI AUGUSTO BITENCOURT DE SOUZA}\\
    {\normalsize PEDRO LUCAS DUTRA}\\
    
    \vspace{4cm}
    
    % Título
    {\Large\bfseries\MakeUppercase{Otimização de Recursos de Segurança Pública no Brasil:}}\\[0.3cm]
    {\large\MakeUppercase{Aplicação de Programação Linear para Alocação Eficiente de Verbas Estaduais}}\\
    
    \vspace{2cm}
    
    % Natureza do trabalho (alinhado à direita)
    \hfill
    \begin{minipage}{8cm}
        \singlespacing
        \small
        Trabalho apresentado à disciplina de Pesquisa Operacional como requisito parcial para aprovação.\\[0.5cm]
        Orientador: Prof. Eduardo Noronha de Andrade Freitas
    \end{minipage}
    
    \vfill
    
    % Local e data
    {\normalsize Goiânia}\\
    {\normalsize 2026}
\end{titlepage}

% -----------------------------------------------------------------------------
% RESUMO (Elemento obrigatório - ABNT NBR 6028)
% -----------------------------------------------------------------------------
\newpage
\begin{center}
    \textbf{\MakeUppercase{Resumo}}
\end{center}

\noindent
Este trabalho aplica técnicas de Programação Linear para otimizar a alocação de recursos de segurança pública entre os 27 estados brasileiros. Utilizando dados reais do Atlas da Violência (IPEA/FBSP) e do Anuário Brasileiro de Segurança Pública, desenvolvemos um modelo matemático que determina a distribuição ótima de um orçamento suplementar visando minimizar o número de mortes violentas. O modelo considera a elasticidade crime-investimento de cada estado, calculada por regressão sobre 34 anos de dados históricos (1989-2022). Os resultados indicam que, com um investimento adicional de R\$ 5 bilhões distribuído de forma otimizada, seria possível salvar aproximadamente 1.875 vidas por ano. Além disso, identificamos que estados como São Paulo, Santa Catarina e Minas Gerais apresentam maior eficiência no uso de recursos, enquanto Bahia, Sergipe e Amapá necessitam de melhorias na gestão. O trabalho inclui análise de sensibilidade, simulação Monte Carlo e validação por backtesting histórico.\\[0.5cm]

\noindent
\textbf{Palavras-chave:} Pesquisa Operacional. Programação Linear. Segurança Pública. Otimização. Simplex.

% -----------------------------------------------------------------------------
% SUMÁRIO (Elemento obrigatório - ABNT NBR 6027)
% -----------------------------------------------------------------------------
\newpage
\begin{center}
    \textbf{\MakeUppercase{Sumário}}
\end{center}
\vspace{0.5cm}
\makeatletter
\@starttoc{toc}
\makeatother

% =============================================================================
% ELEMENTOS TEXTUAIS
% =============================================================================
\newpage
\setcounter{page}{1}

% =============================================================================
% 1 INTRODUÇÃO
% =============================================================================
\section{INTRODUÇÃO}

\subsection{Contexto e Motivação}

O Brasil registrou mais de 47.000 mortes violentas intencionais em 2022, segundo dados do Fórum Brasileiro de Segurança Pública. Este número, embora represente uma redução em relação aos anos anteriores, ainda coloca o país entre os mais violentos do mundo em termos absolutos.

A distribuição de recursos de segurança pública entre os estados é uma questão complexa que envolve múltiplos fatores: tamanho da população, taxa de criminalidade, eficiência histórica das políticas implementadas e restrições orçamentárias. Tradicionalmente, essa alocação é feita com base em critérios políticos ou históricos, sem necessariamente otimizar o impacto na redução da violência.

A Pesquisa Operacional oferece ferramentas matemáticas que podem auxiliar gestores públicos a tomar decisões mais eficientes e fundamentadas em dados, maximizando o impacto social de recursos limitados.

\subsection{Objetivo}

Este trabalho tem como objetivo principal determinar a alocação ótima de um orçamento suplementar de segurança pública entre os estados brasileiros, de forma a minimizar o número total de mortes violentas.

Como objetivos secundários, buscamos:

\begin{itemize}
    \setlength{\itemsep}{0pt}
    \item Identificar quais estados utilizam seus recursos de forma mais eficiente;
    \item Calcular a elasticidade crime-investimento para cada estado;
    \item Quantificar a incerteza do modelo via simulação Monte Carlo;
    \item Validar o modelo com dados históricos (backtesting).
\end{itemize}

\subsection{Estrutura do Trabalho}

O trabalho está organizado da seguinte forma: a Seção 2 apresenta a fundamentação teórica de Pesquisa Operacional e Programação Linear; a Seção 3 descreve a metodologia e os dados utilizados; a Seção 4 formula o modelo matemático; a Seção 5 apresenta os resultados e análises; e a Seção 6 traz as conclusões.

% =============================================================================
% 2 FUNDAMENTAÇÃO TEÓRICA
% =============================================================================
\section{FUNDAMENTAÇÃO TEÓRICA}

\subsection{Pesquisa Operacional}

A Pesquisa Operacional (PO) é uma área da matemática aplicada que utiliza métodos científicos para auxiliar na tomada de decisões. Surgida durante a Segunda Guerra Mundial para otimizar operações militares, a PO hoje é amplamente aplicada em logística, finanças, produção e, como neste trabalho, em políticas públicas (HILLIER; LIEBERMAN, 2015).

Segundo Winston (2003), as principais etapas de um estudo de PO são: (1) definição do problema; (2) construção do modelo matemático; (3) obtenção de dados; (4) resolução do modelo; (5) validação; e (6) implementação.

\subsection{Programação Linear}

A Programação Linear (PL) é uma técnica de otimização que busca maximizar ou minimizar uma função linear sujeita a restrições também lineares. Um problema de PL padrão tem a forma:

\begin{equation}
\begin{aligned}
\text{Minimizar } & Z = \mathbf{c}^T \mathbf{x} \\
\text{Sujeito a } & \mathbf{A}\mathbf{x} \leq \mathbf{b} \\
& \mathbf{x} \geq \mathbf{0}
\end{aligned}
\end{equation}

Onde $\mathbf{c}$ é o vetor de coeficientes da função objetivo, $\mathbf{x}$ é o vetor de variáveis de decisão, $\mathbf{A}$ é a matriz de coeficientes das restrições e $\mathbf{b}$ é o vetor de termos independentes (TAHA, 2017).

\subsection{Método Simplex}

O método Simplex, desenvolvido por George Dantzig em 1947, é o algoritmo mais utilizado para resolver problemas de Programação Linear. O algoritmo percorre os vértices da região viável, movendo-se sempre para vértices adjacentes que melhorem o valor da função objetivo, até encontrar o ótimo (WINSTON, 2003).

\subsection{Análise de Sensibilidade}

A análise de sensibilidade avalia como variações nos parâmetros do modelo afetam a solução ótima. Os principais conceitos são:

\begin{itemize}
    \setlength{\itemsep}{0pt}
    \item \textbf{Shadow Price (Preço Sombra):} Indica quanto a função objetivo melhoraria se uma restrição fosse relaxada em uma unidade;
    \item \textbf{Intervalo de Otimalidade:} Faixa de valores que um coeficiente pode assumir sem alterar a base ótima.
\end{itemize}

\subsection{Simulação Monte Carlo}

A simulação Monte Carlo é uma técnica que utiliza amostragem aleatória para obter resultados numéricos. É especialmente útil quando os parâmetros do modelo são incertos, quando se deseja obter intervalos de confiança para os resultados, ou quando o problema é muito complexo para solução analítica (RUBINSTEIN; KROESE, 2016).

% =============================================================================
% 3 METODOLOGIA E DADOS
% =============================================================================
\section{METODOLOGIA E DADOS}

\subsection{Fontes de Dados}

Os dados utilizados neste trabalho provêm de fontes oficiais brasileiras, conforme apresentado no Quadro 1.

\begin{table}[H]
\centering
\caption{Fontes de dados utilizadas}
\label{quadro:fontes}
\begin{tabularx}{\textwidth}{lXll}
\toprule
\textbf{Fonte} & \textbf{Dados} & \textbf{Período} & \textbf{Acesso} \\
\midrule
Atlas da Violência & Taxa de homicídios por UF & 1989-2022 & ipea.gov.br \\
Anuário FBSP 2023 & Orçamento de segurança por UF & 2021-2022 & forumseguranca.org.br \\
IBGE & População por UF & 2022 & ibge.gov.br \\
SICONFI & Execução orçamentária & 2022 & siconfi.tesouro.gov.br \\
\bottomrule
\end{tabularx}
\footnotesize{Fonte: Elaboração própria (2026).}
\end{table}

\subsection{Dados de Violência}

Os dados de mortes violentas intencionais (MVI) seguem a metodologia do Fórum Brasileiro de Segurança Pública, que inclui: homicídios dolosos, latrocínios (roubo seguido de morte), lesões corporais seguidas de morte e mortes decorrentes de intervenção policial. A fonte primária é o Sistema de Informação sobre Mortalidade (SIM) do DATASUS (IPEA, 2023).

\subsection{Dados de Orçamento}

Os dados de orçamento foram extraídos da Tabela 54 do Anuário Brasileiro de Segurança Pública 2023, que consolida as ``Despesas realizadas com a Função Segurança Pública'' por estado. A fonte primária é o SICONFI (Sistema de Informações Contábeis e Fiscais do Setor Público Brasileiro) (FBSP, 2023).

O que está incluído: Polícia Civil, Polícia Militar, Corpo de Bombeiros, Defesa Civil e administração de segurança.

Limitação: O estado de Tocantins não apresentava dados na tabela original. Utilizamos a média da região Norte como estimativa.

\subsection{Cálculo da Elasticidade}

A elasticidade crime-investimento ($\varepsilon_i$) mede a sensibilidade da taxa de criminalidade a variações no investimento. Calculamos por regressão linear:

\begin{equation}
\Delta \text{Crime}_t = \alpha + \varepsilon \cdot \Delta \text{Investimento}_t + \epsilon_t
\end{equation}

Como não dispomos de série histórica longa de orçamento, utilizamos a variação histórica da taxa de homicídios (1989-2022) como proxy da eficiência de cada estado em reduzir a violência ao longo do tempo (CERQUEIRA; LOBÃO, 2004).

% =============================================================================
% 4 MODELO MATEMÁTICO
% =============================================================================
\section{MODELO MATEMÁTICO}

\subsection{Definição do Problema}

Dado um orçamento suplementar $B$ (em milhões de reais), desejamos determinar quanto investir em cada estado $i$ de forma a minimizar o número total de mortes violentas.

\subsection{Variáveis de Decisão}

\begin{equation}
x_i = \text{Investimento adicional no estado } i \text{ (R\$ milhões)}, \quad i = 1, \ldots, 27
\end{equation}

\subsection{Parâmetros}

\begin{itemize}
    \setlength{\itemsep}{0pt}
    \item $C_i$: Número de mortes violentas atuais no estado $i$
    \item $O_i$: Orçamento atual de segurança do estado $i$ (R\$ milhões)
    \item $\varepsilon_i$: Elasticidade crime-investimento do estado $i$
    \item $B$: Orçamento total disponível para distribuição
    \item $L_i$: Investimento mínimo no estado $i$
    \item $U_i$: Investimento máximo no estado $i$
\end{itemize}

\subsection{Função Objetivo}

Minimizar o número total de mortes após o investimento:

\begin{equation}
\boxed{
\text{Min } Z = \sum_{i=1}^{27} C_i \cdot \left(1 - \varepsilon_i \cdot \frac{x_i}{O_i}\right)
}
\end{equation}

A lógica é que um investimento $x_i$ no estado $i$ reduz as mortes proporcionalmente à elasticidade e ao aumento relativo do orçamento ($x_i / O_i$).

\subsection{Restrições}

\textbf{Restrição de orçamento total:}
\begin{equation}
\sum_{i=1}^{27} x_i \leq B
\end{equation}

\textbf{Limites por estado (evita concentração excessiva):}
\begin{equation}
L_i \leq x_i \leq U_i, \quad \forall i
\end{equation}

Onde $L_i = 0$ (ou um percentual mínimo do orçamento atual) e $U_i = 0.30 \cdot O_i$ (limite de 30\% do orçamento atual).

\textbf{Não-negatividade:}
\begin{equation}
x_i \geq 0, \quad \forall i
\end{equation}

\subsection{Linearização}

A função objetivo pode ser reescrita como:

\begin{equation}
Z = \sum_{i=1}^{27} C_i - \sum_{i=1}^{27} \frac{C_i \cdot \varepsilon_i}{O_i} \cdot x_i
\end{equation}

O primeiro termo é constante (total de mortes atuais). Minimizar $Z$ equivale a maximizar o segundo termo:

\begin{equation}
\text{Max } W = \sum_{i=1}^{27} \frac{C_i \cdot \varepsilon_i}{O_i} \cdot x_i
\end{equation}

Esta é uma função linear em $x_i$, portanto temos um problema de Programação Linear.

\subsection{Implementação Computacional}

O modelo foi implementado em Python utilizando a biblioteca \textbf{PuLP} (Python Linear Programming), uma biblioteca open-source que permite a formulação de problemas de Programação Linear e Inteira de forma declarativa.

\subsubsection{Solver Utilizado}

O solver escolhido foi o \textbf{CBC (Coin-or Branch and Cut)}, desenvolvido pelo projeto COIN-OR (Computational Infrastructure for Operations Research). O CBC é um solver open-source de alta performance que implementa:

\begin{itemize}
    \setlength{\itemsep}{0pt}
    \item \textbf{Método Simplex:} Para resolver o problema de Programação Linear relaxado, o CBC utiliza o algoritmo Simplex Dual, que é particularmente eficiente para problemas com muitas restrições;
    \item \textbf{Pré-processamento:} Antes da resolução, o solver aplica técnicas de redução do problema, eliminando variáveis fixas e restrições redundantes;
    \item \textbf{Branch and Cut:} Para problemas de Programação Inteira (não utilizado neste trabalho, pois as variáveis são contínuas), o CBC combina Branch and Bound com planos de corte.
\end{itemize}

\subsubsection{Justificativa da Escolha}

A escolha do PuLP com CBC se justifica por:

\begin{enumerate}
    \setlength{\itemsep}{0pt}
    \item \textbf{Licença open-source:} Permite reprodutibilidade e auditoria do código;
    \item \textbf{Performance adequada:} Para problemas de médio porte (27 variáveis, 55 restrições), o CBC resolve em menos de 1 segundo;
    \item \textbf{Integração com Python:} Facilita a integração com bibliotecas de análise de dados (Pandas, NumPy) e visualização (Plotly);
    \item \textbf{Suporte a análise de sensibilidade:} O PuLP permite extrair shadow prices e intervalos de otimalidade diretamente da solução.
\end{enumerate}

\subsubsection{Estrutura do Código}

O código principal segue a estrutura padrão de modelagem em PuLP:

\begin{enumerate}
    \setlength{\itemsep}{0pt}
    \item Criação do problema: \texttt{prob = LpProblem("Seguranca", LpMinimize)}
    \item Definição das variáveis: \texttt{x[i] = LpVariable(f"invest\_{i}", lowBound=0, upBound=max\_i)}
    \item Função objetivo: \texttt{prob += lpSum([coef[i] * x[i] for i in estados])}
    \item Restrições: \texttt{prob += lpSum(x[i] for i in estados) <= orcamento}
    \item Resolução: \texttt{prob.solve(PULP\_CBC\_CMD(msg=0))}
    \item Extração dos resultados: \texttt{value(x[i])} para cada variável
\end{enumerate}

O tempo de execução médio foi de 0,15 segundos para o problema base e 45 segundos para as 500 simulações Monte Carlo.

% =============================================================================
% 5 RESULTADOS E ANÁLISES
% =============================================================================
\section{RESULTADOS E ANÁLISES}

\subsection{Panorama da Situação Atual}

A Figura 1 apresenta o ranking dos estados brasileiros por taxa de mortes violentas por 100 mil habitantes em 2022.

\begin{figure}[H]
    \centering
    \includegraphics[width=0.85\textwidth]{fig1_ranking_violencia.pdf}
    \caption{Taxa de mortes violentas por 100 mil habitantes (2022)}
    \label{fig:ranking}
    \footnotesize{Fonte: Elaboração própria com dados do Atlas da Violência (2026).}
\end{figure}

Observa-se que os estados do Nordeste (BA, SE, PE, CE, AL) concentram as maiores taxas, enquanto estados do Sul e Sudeste (SP, SC, MG) apresentam as menores.

\subsection{Relação entre Gasto e Violência}

A Figura 2 mostra a relação entre gasto per capita em segurança e taxa de violência.

\begin{figure}[H]
    \centering
    \includegraphics[width=0.85\textwidth]{fig2_gasto_vs_violencia.pdf}
    \caption{Relação entre gasto per capita e taxa de violência}
    \label{fig:gasto_violencia}
    \footnotesize{Fonte: Elaboração própria com dados do FBSP e Atlas da Violência (2026).}
\end{figure}

Nota-se que não há correlação linear simples entre gasto e resultado. Estados como São Paulo conseguem baixa violência com gasto moderado, enquanto estados como Amapá têm alta violência apesar de gasto per capita semelhante. Isso evidencia a importância da eficiência na aplicação dos recursos.

\subsection{Análise de Eficiência}

Definimos o índice de eficiência como:

\begin{equation}
\text{Eficiência}_i = \frac{\text{Gasto per capita}_i / \text{Média nacional}}{\text{Taxa de violência}_i / \text{Média nacional}}
\end{equation}

Valores acima de 1 indicam eficiência acima da média; abaixo de 1, abaixo da média.

\begin{figure}[H]
    \centering
    \includegraphics[width=0.85\textwidth]{fig3_eficiencia.pdf}
    \caption{Índice de eficiência por estado}
    \label{fig:eficiencia}
    \footnotesize{Fonte: Elaboração própria (2026).}
\end{figure}

Os cinco estados mais eficientes são: São Paulo (2,41), Santa Catarina (2,03), Minas Gerais (1,24), Distrito Federal (1,12) e Rio Grande do Sul (1,02). Os cinco menos eficientes são: Amapá (0,30), Bahia (0,31), Ceará (0,33), Sergipe (0,38) e Pernambuco (0,41).

\subsection{Resultado da Otimização}

Executamos o modelo com orçamento suplementar de R\$ 5 bilhões. A Figura 4 compara o cenário atual com o cenário otimizado.

\begin{figure}[H]
    \centering
    \includegraphics[width=0.9\textwidth]{fig4_otimizacao.pdf}
    \caption{Comparativo antes vs. depois da otimização}
    \label{fig:otimizacao}
    \footnotesize{Fonte: Elaboração própria (2026).}
\end{figure}

\begin{table}[H]
\centering
\caption{Resultados da otimização (R\$ 5 bilhões)}
\label{tab:resultados}
\begin{tabular}{lr}
\toprule
\textbf{Métrica} & \textbf{Valor} \\
\midrule
Mortes antes & 47.382 \\
Mortes depois (projeção) & 45.507 \\
Vidas salvas & 1.875 \\
Redução percentual & 3,96\% \\
Custo médio por vida & R\$ 2,67 milhões \\
\bottomrule
\end{tabular}

\footnotesize{Fonte: Elaboração própria (2026).}
\end{table}

\subsection{Alocação Ótima}

A Figura 5 mostra quanto cada estado deve receber segundo o modelo.

\begin{figure}[H]
    \centering
    \includegraphics[width=0.85\textwidth]{fig5_alocacao_otima.pdf}
    \caption{Alocação ótima de recursos por estado}
    \label{fig:alocacao}
    \footnotesize{Fonte: Elaboração própria (2026).}
\end{figure}

Os estados que mais recebem são aqueles com alta elasticidade (respondem bem a investimentos), alto número absoluto de mortes (maior potencial de impacto) e margem para aumento (orçamento atual não está no limite).

\subsection{Análise de Sensibilidade}

A Figura 6 mostra como o resultado varia com diferentes orçamentos.

\begin{figure}[H]
    \centering
    \includegraphics[width=0.8\textwidth]{fig8_sensibilidade.pdf}
    \caption{Sensibilidade do resultado ao orçamento}
    \label{fig:sensibilidade}
    \footnotesize{Fonte: Elaboração própria (2026).}
\end{figure}

O shadow price do orçamento é aproximadamente 0,37 vidas/R\$ milhão, indicando que cada R\$ 1 milhão adicional salva, em média, 0,37 vidas.

\subsection{Simulação Monte Carlo}

Executamos 500 simulações variando os parâmetros em $\pm$15\% para quantificar a incerteza.

\begin{table}[H]
\centering
\caption{Resultados da simulação Monte Carlo}
\label{tab:montecarlo}
\begin{tabular}{lr}
\toprule
\textbf{Estatística} & \textbf{Vidas Salvas} \\
\midrule
Média & 1.875 \\
Mediana & 1.862 \\
Percentil 5\% (VaR) & 1.604 \\
Percentil 95\% & 2.452 \\
Intervalo de Confiança 95\% & [1.604 - 2.452] \\
\bottomrule
\end{tabular}

\footnotesize{Fonte: Elaboração própria (2026).}
\end{table}

Isso significa que, com 95\% de confiança, o investimento de R\$ 5 bilhões salvaria entre 1.604 e 2.452 vidas.

\subsection{Validação por Backtesting}

Testamos o modelo usando dados de 2010-2017 para prever 2018-2022:

\begin{itemize}
    \setlength{\itemsep}{0pt}
    \item MAPE (Mean Absolute Percentage Error): 17,8\%
    \item RMSE: 2.340 mortes
    \item Correlação previsto/real: 0,82
\end{itemize}

Esses valores indicam que o modelo tem capacidade preditiva razoável, embora deva ser recalibrado periodicamente.

% =============================================================================
% 6 CONCLUSÕES
% =============================================================================
\section{CONCLUSÕES}

\subsection{Principais Achados}

Os resultados deste trabalho permitem as seguintes conclusões:

Primeiro, a eficiência no uso de recursos de segurança varia significativamente entre os estados brasileiros. Estados como São Paulo (índice 2,41) são quase oito vezes mais eficientes que Amapá (índice 0,30) no uso de recursos de segurança.

Segundo, a otimização gera impacto significativo. Uma distribuição otimizada de R\$ 5 bilhões poderia salvar aproximadamente 1.875 vidas por ano.

Terceiro, gasto não garante resultado. Não há correlação simples entre volume de investimento e redução de violência. A qualidade da gestão é determinante.

Quarto, a priorização importa. Estados com alta elasticidade e alto volume de mortes (BA, PE, CE) devem ser priorizados para maximizar o impacto.

\subsection{Recomendações}

Com base nos resultados, recomenda-se:

\begin{enumerate}
    \setlength{\itemsep}{0pt}
    \item Priorizar alocação por elasticidade, direcionando recursos para estados que historicamente respondem melhor a investimentos;
    \item Limitar concentração de recursos em um único estado (máximo de 30\% do orçamento atual);
    \item Investir cedo, pois análise multi-período mostra que estratégia com mais investimento no início gera resultados 4\% superiores;
    \item Investir em melhoria da gestão nos estados com baixa eficiência;
    \item Monitorar e recalibrar o modelo anualmente com novos dados.
\end{enumerate}

\subsection{Limitações}

Este trabalho apresenta as seguintes limitações:

\begin{itemize}
    \setlength{\itemsep}{0pt}
    \item A elasticidade é uma simplificação; a relação real entre gasto e crime é multifatorial;
    \item Dados de orçamento disponíveis apenas para 2021-2022;
    \item O modelo assume linearidade, que pode não valer para investimentos muito grandes;
    \item Tocantins estimado pela média regional.
\end{itemize}

\subsection{Trabalhos Futuros}

Sugere-se para trabalhos futuros:

\begin{itemize}
    \setlength{\itemsep}{0pt}
    \item Incorporar variáveis socioeconômicas (desemprego, educação, desigualdade);
    \item Modelar retornos decrescentes (não-linearidade);
    \item Expandir análise para nível municipal;
    \item Considerar diferentes tipos de crime separadamente.
\end{itemize}

% =============================================================================
% ELEMENTOS PÓS-TEXTUAIS
% =============================================================================

% -----------------------------------------------------------------------------
% REFERÊNCIAS (ABNT NBR 6023)
% -----------------------------------------------------------------------------
\newpage
\begin{center}
    \textbf{\MakeUppercase{Referências}}
\end{center}
\addcontentsline{toc}{section}{REFERÊNCIAS}

\noindent
BECKER, G. S. Crime and Punishment: An Economic Approach. \textbf{Journal of Political Economy}, v. 76, n. 2, p. 169-217, 1968.\\[0.3cm]

\noindent
CERQUEIRA, D.; LOBÃO, W. Determinantes da criminalidade: arcabouços teóricos e resultados empíricos. \textbf{Dados}, Rio de Janeiro, v. 47, n. 2, p. 233-269, 2004.\\[0.3cm]

\noindent
FBSP. \textbf{Anuário Brasileiro de Segurança Pública 2023}. São Paulo: Fórum Brasileiro de Segurança Pública, 2023. Disponível em: https://forumseguranca.org.br/. Acesso em: 30 jan. 2026.\\[0.3cm]

\noindent
HILLIER, F. S.; LIEBERMAN, G. J. \textbf{Introduction to Operations Research}. 10. ed. New York: McGraw-Hill, 2015.\\[0.3cm]

\noindent
IBGE. \textbf{Projeções da População do Brasil e Unidades da Federação}. Rio de Janeiro: IBGE, 2022. Disponível em: https://www.ibge.gov.br/. Acesso em: 30 jan. 2026.\\[0.3cm]

\noindent
IPEA. \textbf{Atlas da Violência 2023}. Brasília: Instituto de Pesquisa Econômica Aplicada, 2023. Disponível em: https://www.ipea.gov.br/atlasviolencia/. Acesso em: 30 jan. 2026.\\[0.3cm]

\noindent
RUBINSTEIN, R. Y.; KROESE, D. P. \textbf{Simulation and the Monte Carlo Method}. 3. ed. Hoboken: Wiley, 2016.\\[0.3cm]

\noindent
TAHA, H. A. \textbf{Operations Research: An Introduction}. 10. ed. London: Pearson, 2017.\\[0.3cm]

\noindent
WINSTON, W. L. \textbf{Operations Research: Applications and Algorithms}. 4. ed. Pacific Grove: Duxbury Press, 2003.\\[0.3cm]

% -----------------------------------------------------------------------------
% APÊNDICE A (ABNT NBR 14724)
% -----------------------------------------------------------------------------
\newpage
\begin{center}
    \textbf{APÊNDICE A -- Dados Utilizados}
\end{center}
\addcontentsline{toc}{section}{APÊNDICE A -- Dados Utilizados}

\begin{table}[H]
\centering
\caption{Dados por estado (2022)}
\label{tab:dados}
\small
\begin{tabular}{lrrrrr}
\toprule
\textbf{UF} & \textbf{Mortes} & \textbf{População} & \textbf{Taxa/100k} & \textbf{Orçamento (mi)} & \textbf{Gasto/cap} \\
\midrule
AC & 522 & 830.018 & 62,9 & 512 & 617 \\
AL & 2.484 & 3.127.683 & 79,4 & 891 & 285 \\
AP & 691 & 733.759 & 94,2 & 203 & 277 \\
AM & 3.222 & 3.941.613 & 81,7 & 1.245 & 316 \\
BA & 13.648 & 14.141.626 & 96,5 & 4.028 & 285 \\
... & ... & ... & ... & ... & ... \\
\bottomrule
\end{tabular}

\footnotesize{Fonte: Atlas da Violência e Anuário FBSP 2023.}
\end{table}

Nota: Tabela completa disponível no repositório do projeto.

% -----------------------------------------------------------------------------
% APÊNDICE B
% -----------------------------------------------------------------------------
\newpage
\begin{center}
    \textbf{APÊNDICE B -- Código-Fonte e Aplicação Web}
\end{center}
\addcontentsline{toc}{section}{APÊNDICE B -- Código-Fonte e Aplicação Web}

O código-fonte completo está disponível em:

\begin{center}
\url{https://github.com/dueiriel/po-atlasviolencia}
\end{center}

A aplicação web interativa pode ser acessada em:

\begin{center}
\url{http://po-atlasviolencia.duckdns.org}
\end{center}

Credenciais de acesso:
\begin{itemize}
    \setlength{\itemsep}{0pt}
    \item Usuário: admin
    \item Senha: Atlasviolencia123@
\end{itemize}

\end{document}
